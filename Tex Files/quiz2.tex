\documentclass{article}
\usepackage{hyperref}
\usepackage{xcolor}

\usepackage{multicol}
\usepackage{bidipoem}
\usepackage{shapepar}
\usepackage{xepersian}
\settextfont{B Zar}
\defpersianfont\Nas[Scale=1.5,Color=red]{IranNastaliq}
\defpersianfont\green[Color=green]{B Zar}
\title{\Nas آموزش لاتک}
\author{\green\underline{نویسنده}}
\date{\today}
\begin{document}
	\maketitle
	\section{مقدمه}\label{a}
	در بخش
	\ref{a}
	 داریم:
	 \begin{itemize}
	 	\item
	 	فهرست های به هم ریخته
	 	\begin{enumerate}
	 		\item اول
	 		\item two\LTRfootnote{دو}
	 	\end{enumerate}
	 	
	 	\item
	 	فهرست های شمارشی
	 	\begin{description}
	 		\item[اول]	 		
 		 	\end{description}

	 	فهرست های توضیحی
	 	\footnote{2}
	 	\item
	 	فهرست های درون خطی
	 \end{itemize}
 	\begin{traditionalpoem}

 بی تو مهتاب شبی از آن کوچه گذشتم!
&&
 	 همه تن چشم شدم خيره به دنبال تو گشتم.
 \end{traditionalpoem}
	\shapepar{\starshape}
		بی تو مهتاب شبی از آن کوچه گذشتم! همه تن چشم شدم خيره به دنبال تو گشتم.
	
 
\end{document}