\documentclass[openany,12pt]{book}
\usepackage{graphicx}
\usepackage{makeidx}
\makeindex
\usepackage{rotating}
\usepackage{multicol}
\usepackage{multirow}
\usepackage{hyperref}
\usepackage{xepersian}
\settextfont{B Zar}
\title{آموزش لاتک}
\author{نگارنده}
\date{28 مهر 1397}
\begin{document}
	\maketitle
	\tableofcontents
	\listoffigures
	\listoftables
	\chapter{آموزش لاتک}
	\section{جدول و تصاویر}
					جدول 
					\index{جدول}
	\ref{t1}
	و شکل 
%	\ref{f2}
	را رسم کنید.
	
	\begin{table}[h]
		\caption{جدول آزمایشی}
		\begin{center}
			\begin{tabular}{|c|c|c|}
				\hline
				\multirow{2}{*}{ردیف} &  \multicolumn{2}{c|}{مشخصات فردی} \\
				\cline{2-3}
				& نام & نام  خانوادگی \\
				\hline
				\multicolumn{2}{|c}{آزمایش} &    تست  \\
				\hline
			\end{tabular}\label{t1}
		\end{center}	
	\end{table}

	کد زیر را بنویسید.
	
	\begin{latin}
		\begin{verbatim}
		function[c,ceq]=confun(x)
		global n
		global d
		c=[];
		for k=1 : n+1
		ceq (k=sum(d(:,k)*x(1:n+1))-x(k)*x(n+1+k)+x(k)+(x(n+1+k)^2);
		end
		ceq (n+2)=x(1)-1;
		end
		\end{verbatim}
	\end{latin}
\fbox{
	\begin{minipage}[c]{10cm}
	راهنمایی:
	\begin{itemize}
		\item 
		اين فايل را با رده نوشتاری زير توليد کنيد:
	\end{itemize}
\begin{latin}
	$\backslash$documentclass[openany,12pt]{book}
\end{latin}
\begin{itemize}
	\item 
	این متن راهنما را در محیط
\end{itemize}
\begin{latin}
$\backslash$fbox\{


$\backslash$begin{minipage}[c]{10cm}
\end{latin}
متن راهنما
\begin{latin}
		$\backslash$end{minipage}
\}
		\end{latin}

\begin{itemize}
	\item 
	از 
	\lr{pdf4}
	یا کتاب های 
	\ref{b1}
	و
	\ref{b2}
	برای راهنمایی استفاده کنید.
\end{itemize}
\end{minipage}
}

\begin{figure}
	\caption{
		شکل
		\ref{p1}
		شکل
		\ref{p2}
		شکل
		\ref{p3}
		شکل
		\ref{p4}
	} 
	\begin{center}
		
		\begin{figure}[options]
		\centering
		\subfigure[ }[توضيح زير عکس اول
	\includegraphics[options]{image 1}
\label{ {{برچسب عکس اول
		\hspace*{20mm}
		
		
		\includegraphics[height=50pt, angle=-30]{git.jpg}
		\caption{چیز}
		\label{p1}
		\includegraphics[height=50pt, angle=50]{git.jpg}
		\caption{چیز}
		\label{p2}
		\includegraphics[height=50pt, angle=150]{git.jpg}
		\caption{چیز}
		\label{p3}
		
		\includegraphics[height=50pt, angle=230]{git.jpg}
		\caption{چیز}
		\label{p4}
	\end{center}
\end{figure}
\begin{thebibliography}{9}
\bibitem{citekey1}
مشخصات منبع فارسی اول
.
\begin{LTRbibitems}
\resetlatinfont
\bibitem{citekey2}
\lr{reference 2}
.
\end{LTRbibitems}
\end{thebibliography}

\printindex

\end{document}