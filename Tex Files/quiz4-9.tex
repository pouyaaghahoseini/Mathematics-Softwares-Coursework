\documentclass{book}
\usepackage{hyperref}

\usepackage{multicol}
\usepackage{multirow}
\usepackage{xepersian}

\settextfont{B Zar}

\title{آموزش لاتک}
\author{پویا آقاحسینی}
\date{\today}

\begin{document}
	\maketitle
	\tableofcontents
	\listoffigures
	\listoftables
	\chapter{اموزش لاتک}\label{ch1}
	\section{جدول و تصاویر}
	چدول 
	\ref{جدول اول}
	وشکل   را رسم کنید.
	\begin{table}[h!]
		\caption{جدول آزمایشی}
		\label{جدول اول}
		\begin{center}
			\begin{tabular}{|c|c|c|}
				\hline
				\multirow{2}{*}{ردیف}&
				\multicolumn{2}{|c|}
				{مشخصات فردی}
				\\ \cline{2-3}
				&  نام
				&نام خانوادگی
				\\
				\hline
				\multicolumn{2}{|c||}
				{ازمایش}&
				تست
				\\ 
				\hline
			\end{tabular}
		\end{center}
	\end{table}
برای وارد کردن يک واژه از دستور
	\begin{latin}
		glspl
		\end{latin}
	بايد استفاده نمود. مثل واژه متغيرهای تصادفی
	 \footnote{\label{f1}\lr{Random Variable}}
	  که اگر در فايل تک آن نگاه کنید، مشاهده می کنید که برای وارد کردن واژه متغیر های تصادفی از دستور 
	  \lr{glspl}
	  استفاده شده است. در ضمن در اولين استفاده از اين واژه، معادل انگليسی آن نيز پاورقی خورده است. و اکنون واژه
	   	 \footnote{\label{f2}\lr{Optimization}}
	    را تعريف می کنيم.
	    از اخصارات ، اختصارات 
	    \lr{BAN\footnote{\label{f3}\lr{Body Area Network}}}
	    و
	    \lr{CDMA\footnote{\label{f4}\lr{Code Division Multiplexing Access}}}
	    را وارد می کنيم. برای بار اول پاورقی می خورد. اما برای
	    بار دوم پاورقي زده نمي شود. اگر واژه و يا اختصاري را در متن با دستورات
	    \lr{gls}
	    و
	    \lr{glspl}
	    وارد نکنید ، واژه نه
	    در متن ظاهر شده و نه در واژ هنامه ها وارد می شود.
\end{document}