\documentclass{book}
\usepackage{multicol}
\usepackage{multirow}
\usepackage{xepersian}
\settextfont{Arial}
\author{نویسنده}
\title{اموزش لاتک}
\date{5اذر1397}
\begin{document}
	\maketitle
	\tableofcontents
	\listoffigures
	\listoftables
	\chapter{اموزش لاتک}
	\section{جدول و تصاویر}
	چدول 
	\ref{جدول یک}
	وشکل   را رسم کنید.
	\begin{table}
		\caption{جدول ازمایشی}
		\label{جدول یک}
		\begin{tabular}{|c||c||c|}
			\hline
			\multirow{2}{*}{ردیف}&
			
			\multicolumn{2}{|c|}{مشخصات فردی} 
			\\ \cline{2-3}
			&  نام&نام خانوادگی
			\\
			\hline
			\multicolumn{2}{|c||}{ازمایش}&
			تست
			\\ 
			\hline
		\end{tabular}
	\end{table}
	\\متغیرهای تصادفی
	\LTRfootnote{random variable}{flushright}
	
	واژه ی 
	\LTRfootnote{optimization}
	BAN 
	\LTRfootnote{body area network}
	cdma 
	\LTRfootnote{code division multiplexing access}
\end{document}